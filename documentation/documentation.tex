\documentclass[a4paper, 11pt]{article}
\usepackage{graphicx}
\usepackage{algorithm}
\usepackage{algpseudocode}
\usepackage{amsmath}
%\usepackage[a4paper,top=2cm,bottom=2cm,left=2.4cm,right=2.4cm]{geometry}

\usepackage{listings}

\title { Interactive Graphics Project\\ \bigskip \large IG Sapienza}
\date{22 September 2019}
\author{Marco Morella 1708962, \\Fortunato Tocci 1708962, \\Silvio Dei Giudici 1708962}

\begin{document}

\maketitle

%USER MANUAL PART
\section{User Manual}
%esponi cosa può fare l'utente (non specificare cosa succede internamente perchè questo lo specificherai nell'ultima sezione)
%selezione del personaggio, giorno/notte (anche in game), difficoltà, comandi

%TECHNICAL PART
\section{Development environment}
\subsection{WebGL}
%parla delle cose che stanno nelle slide eventualmente (un riassuntino), tipo vertex e fragment shader, come si rappresentano oggetti (vertici)
%come rappresentare movimenti (con matrici ecc)
\subsection{Library and tools}
%three js, OrbitControls, Browser firefox (uso della console), IDE (se vuoi)

\section{Technical Solutions}
\subsection{Hierarchical models}
%ogni personaggio, auto, tree, ambiente(River, Road) (ricorda che hai track --> tree/auto (sono figli del track))
\subsection{Light and texture}
%parla della luce del sole (con i dettagli della luce scelta), la luce del lampione (con alternanza giorno notte), le texture dei tronchi e delle foglie
\subsection{Animations}
%anche l'animazione dello splash, nella sezione successiva parliamo solo della detection
%animazione degli animali, animazione auto/trunk, animazione sunk, animazione crash
\subsection{Crash, sunk and object detection}
%parla sia del crash con auto, splash del personaggio e detection dei trees/bush
\subsection{Optimizations}
%parla dell'uso dei Layers e delle animazioni che partono solo quando il personaggio si avvicina

\section{User-game interactions}
%esponi le interazioni che l'utente può eseguire specificando cosa succede internamente (ex: ad ogni livello corrispondo 6 livelli per easy ecc)
\end{document}
